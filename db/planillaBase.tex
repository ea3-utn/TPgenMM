%%%%%%%%%%%%%%%%%%%%%%%%%%%%%%%%%%%%%%%%%
% Engineering Calculation Paper
% LaTeX Template
% Version 1.0 (20/1/13)
%
% This template has been downloaded from:
% http://www.LaTeXTemplates.com
%
% Original author:
% Dmitry Volynkin (dim_voly@yahoo.com.au)
%
% License:
% CC BY-NC-SA 3.0 (http://creativecommons.org/licenses/by-nc-sa/3.0/)
%
%%%%%%%%%%%%%%%%%%%%%%%%%%%%%%%%%%%%%%%%%

%----------------------------------------------------------------------------------------
%	PACKAGES AND OTHER DOCUMENT CONFIGURATIONS
%----------------------------------------------------------------------------------------

\documentclass[12pt,a4paper,spanish]{article} % Use A4 paper with a 12pt font size - different paper sizes will require manual recalculation of page margins and border positions

\usepackage[spanish]{babel} % Idioma
\selectlanguage{spanish}
\usepackage{marginnote} % Required for margin notes
\usepackage{wallpaper} % Required to set each page to have a background
\usepackage{lastpage} % Required to print the total number of pages
\usepackage[left=1.3cm,right=4.6cm,top=1.8cm,bottom=4.0cm,marginparwidth=3.4cm]{geometry} % Adjust page margins
\usepackage{amsmath} % Required for equation customization
\usepackage{amssymb} % Required to include mathematical symbols
\usepackage{xcolor} % Required to specify colors by name
\usepackage{physics} % Required to specify colors by name
\usepackage{subcaption}
\usepackage[demo]{graphicx}
\usepackage{enumitem}

\usepackage[hyphens]{url}
\usepackage{hyperref}

\urlstyle{tt}

\newcommand{\MatrizN}[1]{\ensuremath{\mathbf{#1}}}
\newcommand{\Matriz}[1]{\ensuremath{\left[{#1}\right]}}
\newcommand{\VectorN}[1]{\ensuremath{\mathbf{#1}}}
\newcommand{\Vector}[1]{\ensuremath{\left\{{#1}\right\}}}
\newcommand{\dvar}[1]{\ensuremath{\delta{#1}}}
\newcommand{\vol}{\ensuremath{\text{Vol}}}
\newcommand{\dvol}{\ensuremath{\text{dVol}}}
\newcommand{\area}{\ensuremath{\text{Area}}}
\newcommand{\darea}{\ensuremath{\text{dArea}}}


\usepackage{fancyhdr} % Required to customize headers
\setlength{\headheight}{80pt} % Increase the size of the header to accommodate meta-information
\pagestyle{fancy}\fancyhf{} % Use the custom header specified below
\renewcommand{\headrulewidth}{0pt} % Remove the default horizontal rule under the header

\setlength{\parindent}{0cm} % Remove paragraph indentation
\newcommand{\tab}{\hspace*{2em}} % Defines a new command for some horizontal space
\newcommand{\tabH}{\vspace*{2em}} % Defines a new command for some horizontal space

\newcommand\BackgroundStructure{ % Command to specify the background of each page
\setlength{\unitlength}{1mm} % Set the unit length to millimeters

\setlength\fboxsep{0mm} % Adjusts the distance between the frameboxes and the borderlines
\setlength\fboxrule{0.5mm} % Increase the thickness of the border line
\put(10, 10){\fcolorbox{black}{green!10}{\framebox(192,247){}}} % Main content box
% \put(165, 10){\fcolorbox{black}{blue!10}{\framebox(37,247){}}} % Margin box
\put(10, 262){\fcolorbox{black}{white!10}{\framebox(192, 25){}}} % Header box
% \put(100, 262){\includegraphics[width=102mm,keepaspectratio]{./imagenes/logo}} % Logo box - maximum height/width:
\put(111, 264){\includegraphics[width=85mm,keepaspectratio]{./imagenes/logo}} % Logo box - maximum height/width: 
}

%----------------------------------------------------------------------------------------
%	HEADER INFORMATION
%----------------------------------------------------------------------------------------

\fancyhead[L]{\begin{tabular}{l r} % The header is a table with 4 columns
% \thepage/\pageref{LastPage} \\ % Project name and page count
\textbf{ESTRUCTURAS AERONAUTICAS III} \\
\textbf{Trabajo pr\'actico dos} \\
\textbf{M\'etodo de Rayleigh-Ritz} \\
\textbf{ALUMNO} \\
\end{tabular}}

%----------------------------------------------------------------------------------------

\begin{document}

\bibliographystyle{plain}

\AddToShipoutPicture{\BackgroundStructure} % Set the background of each page to that specified above in the header information section

%----------------------------------------------------------------------------------------
%	DOCUMENT CONTENT
%----------------------------------------------------------------------------------------

\section{Requerimientos generales}

Presentar un \textbf{informe} donde se escriba para cada enunciado:

\begin{enumerate}
\item Las funciones propuestas.
\item La expresi\'on detallada de la energ\'ia de deformaci\'on y el potencial de cargas.
\item Para el problema 2, dibujar un esquema del modelo utilizado detallando apoyos, cargas aplicadas, geometr\'ia, etc.
\item Comparaci\'on de resultados con las funciones de aproximaci\'on propuestas.
\item En caso de corresponder, las respuestas a las preguntas del enunciado.
\item Conclusiones y observaciones sobre los resultados obtenidos.
\end{enumerate}

Adem\'as se requiere presentar todas las \textbf{planillas de c\'alculo} utilizadas para la resoluci\'on del trabajo pr\'actico.

\section{Enunciado}
Se pide para el caso de la figura:

\begin{enumerate}[label=(\alph*)]
\item Obtener la funci\'on aproximada de desplazamientos del siguiente caso, aplicando el m\'etodo de Rayleigh-Ritz, utilizando los datos dados. Obtener las expresiones para 1 y 2 constantes de aproximaci\'on. Expresar en forma expl\'icita ambas funciones.

\item Graficar la respuesta para ambos casos en un mismo dibujo.

\item En caso de ser posible exprese la soluci\'on exacta del problema.

\item Explique si la soluci\'on obtenida es fisicamente correcta y si se la puede considerar como v\'alida desde el punto de vista de la ingenier\'ia.
\end{enumerate}

\tabH
